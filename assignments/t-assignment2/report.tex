\documentclass[12pt,a4paper]{article}
\usepackage[english, science, large]{../template/ku-frontpage}
\usepackage{tabularx}
\usepackage{ltablex}
\usepackage{minted}
\setminted[text]{
frame=lines,
framesep=2mm,
baselinestretch=1.1,
fontsize=\footnotesize,
linenos,
breaklines}
\hypersetup{
    colorlinks=false,
    pdfborder={0 0 0},
}
\begin{document}

\title{ACS Theory Assignment 2}
\subtitle{}

\author{Kai Arne S. Myklebust, Silvan Adrian}
\date{Handed in: \today}
	
\maketitle
\tableofcontents

\section{Question 1: Concurrency Control Concepts}

\subsection{Task 1}

\begin{table}[!htbp]
    \centering
    \begin{tabularx}{\textwidth}{l|l|l}
        \hline
        T1 & T2 & T3 \\ 
        \hline
        R(A) &  &  \\
             & W(A) & \\
             & commit & \\
        W(A) & & \\
        Commit & & \\
        		& & W(A) \\
        		& & Commit \\     
        \hline
    \end{tabularx}
\end{table}

\subsection{Task 2}

\subsection{Task 3}
\subsection{Task 4}

\section{Question 2: Recovery Concepts}
\subsection{Task 1}
We do not need a redo scheme, because all changes of committed transactions are guaranteed to have been written to disk at commit time.
We also don't need the undo scheme, because the changes of those aborted transactions have not been written to disk.

\subsection{Task 2}
The difference of stable storage and non-volatile storage is that stable storage is implemented by maintaining multiple copies of information on non-volatile storage.
Access time on non-volatile storage is therefore much faster than on stable storage.

Non-volatile storage is guaranteed to survive crashes but can be subject to media failures, while stable storage is guaranteed to survive both.


\subsection{Task 3}
The log tail needs to be forced to stable storage in following 2 situations:
\begin{itemize}
	\item When a transaction is committed
	\item After modifying of pages	
\end{itemize}

For the first situation, when a transaction is committed it has be ensured that the record of changes have been written in the log before it's written to disk.
This way we know what changes have been done even after a crash.

For the second situation, when a page gets modified we have to know that there has been a change without committing so that we are able to undo the not committed changes. 

For both situations the durability is sufficient because we are able to undo modifications and ensure that all committed transactions survive a crash.

\section{Question 3: More Concurrency Control}
\subsection{Task 1}
\subsection{Task 2}
\subsection{Task 3}

\section{Question 4: ARIES}

\section{Question 5: More ARIES}

\end{document}}